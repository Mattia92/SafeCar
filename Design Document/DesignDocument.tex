%Type of document
\documentclass[a4paper, 12pt]{report}

%For easy management of document margins and the document page size
\usepackage[right=3.5cm,left=3.5cm,top=3.5cm,bottom=3.5cm]{geometry}

%Allows to insert graphic files within a document
\usepackage{graphicx}

%Sup­ports com­pressed, sorted lists of nu­mer­i­cal ci­ta­tions, and also deals with var­i­ous punc­tu­a­tion and %other is­sues of rep­re­sen­ta­tion, in­clud­ing com­pre­hen­sive man­age­ment of break points
\usepackage{cite}

%It gives LaTeX the possibility to manage links within the document or to any URL when you compile in PDF
\usepackage{hyperref}

%To choose the font encoding of the output text
\usepackage[T1]{fontenc}

%To choose the encoding of the input text
%Consente di usare le lettere accentate
\usepackage[latin1]{inputenc}

%It provides the internationalization of LaTeX. It has to be loaded in any document, and you have to give %as an option the main language you are going to use in the document
\usepackage[english]{babel}

%Allows to write algorithms
\usepackage{algorithm}
\usepackage{algpseudocode}

\normalfont
%Forza LaTeX ad una spaziatura uniforme, invece di lasciare più spazio
%alla fine dei punti fermi come da convenzione inglese
\frenchspacing
%Modifica della spaziatura interlineare
\linespread{1.3}

%Inizia il documento
\begin{document}

%Creazione di un frontespizio personalizzato
\begin{titlepage}

\begin{center}
\Large
\textbf{POLITECNICO DI MILANO} \\
\Large
School of Industrial and Information Engineering \\
Computer Science and Engineering
\end{center}

\addvspace{0.8cm}
%PER INSERIRE IMMAGINE
\begin{figure}[h]
\begin{center}
\includegraphics[width=7cm]{cpt/img/polimi.png}
\end{center}
\end{figure}

\addvspace{0.1cm}
\begin{center}
\LARGE

\textbf{Design and Implementation of Mobile Application Project: Safe Car \\
Design Document}

\end{center}

\addvspace{0.5cm}
\Large
\begin{center}
\begin{tabular}{p{1\textwidth}p{0.3\textwidth}}
Course Professor: Prof. Luciano BARESI \\
\end{tabular}
\end{center}

\addvspace{0.6cm}
\Large
\begin{center}
\begin{tabular}{p{0.6\textwidth}p{0.6\textwidth}}
& Authors: \\
& Mattia	CRIPPA		1039725\\
& Alberto PIROVANO	10396610
\end{tabular}
\end{center}

\vfill
\Large
\begin{center}
Academic Year 2017--2018
\end{center}
\end{titlepage}

\clearpage

\begin{abstract}
SafeCar is an application that has been designed to help the driver in improving its driving style. 
On one side it offers the possibility to inspect the historical user data, and on the other side it provides a hint generation engine during the trip .
The core algorithm of the application communicates in an asynchronous way with a physical object that has to be plugged into the car. This object is the interface between the car and the application, and once connected with it the application is able to access and process the data about the navigation.
During the trip the application analyses these data, processing them to profile the driving style. In addition it generates an index representing a summary of a specific moment of the trip. Based on this index, another algorithm generates a hint message.
This procedure is performed in loop, so the application generates hints continuously during the trip and through these hints the user can obtain some indications about how to improve his driving style.
\end{abstract}

\tableofcontents
\clearpage

\listoffigures
\clearpage

\chapter{Introduction} \label{chap1}
The \textit{Design Document} is a document meant to provide documentation which will be used to help developers in implementing the entire system by providing a general description of the architecture and the design of the system to be built.

\section{Purpose}
The purpose of the Design Document is to provide a description of the system detailed enough to understand which are the components of the system, how they interact, which is their architecture and how they will be deployed. The level of the description is high enough for all the stakeholders to capture the information they need in order to decide whether the system meets their requirements or in order to begin the development work.

\section{Scope}
This document provides a detailed description of \textit{Safe Car} software design and architectural choices. Every portion of the document is designed itself to be comprehensible, but a big picture of the system must be present to the reader in order to obtain the best knowledge on the matter when consulting this document.

\section{Definitions \& Acronysms}
We are going to use a set of specific terms, each one referring to a specific abstract or physical object:

\begin{enumerate}
	\item \underline{Plug}: It is a smart object that the user has to buy and that has to be plugged inside the h732 port of the car. It has to have Bluetooth functionalities enabled
	\item \underline{Trip}: An abstract concept data structure containing all the information about a travel
	\item \underline{Badge}: An abstract object that refers to a specific unlocking condition. It can be locked or unlocked
\end{enumerate}
In the document are often used some technical terms whose definitions are here reported:

\begin{enumerate}
	\item \underline{Layer}: A software level in a software system
	\item \underline{Tier}: An hardware level in a software system
\end{enumerate}
We also need some application specific terminology:

\begin{enumerate}
	\item \underline{DSI}: Driver Safety Index
	\item \underline{ALG1}: High-level hint generation algorithm
	\item \underline{ALG2}: DSI computation engine
	\item \underline{ALG3}: Low-level hint generator agent
\end{enumerate}

\section{Document Structure}
This document is intended for individuals directly involved in the development of Safe Car application. This includes software developers, project consultants, and team managers. This document is not meant to be read sequentially; users are encouraged to jump to any section they find relevant. Below is a brief overview of each part of the document:

\begin{enumerate}
	\item \textbf{Introduction:} This section gives general information about the Design Document of Safe Car application
	\item \textbf{System Overview:} This section contains an overview of the application and its primary functionalities. It also contains assumptions and constraints followed during the design of the software
	\item \textbf{Architectural Design:} This section exposes in details the design chosen for the architecture of the system to be
	\item \textbf{User Interface Design:} This section provides the detailed design information for each component in the current delivery
\end{enumerate}
\clearpage

\chapter{System Overview} \label{chap2}
The aim of our project is to develop an Android application which can be used in the real world.\\
In order to use this application, the driver has to register as a user of the service. The \textit{Registration and Login} procedures can be performed via the custom application functionality or via Google Plus APIs.
The user can change his/her password, drop the account and modify data related to his/her profile whenever he/she wants.
At this point the application flow develops in two different ways:

\begin{itemize}
	\item If the user's smartphone is not in the radio scan area of the Plug, he/she can only inspect user related data about his/her history usage of the application. He can consult the performed trips, his/her profile data or the gained badges
	\item If the user's smartphone is close to the Plug, he/she can pair his/her device with that Plug and, after the procedure has succeeded, he/she can start a new trip. Now the \textit{Driving Experience} actually begins. After this moment, the application will follow the user in his/her driving experience by asking to the plug data about the navigation, for example the acceleration rate or the frequency with which the user is breaking. On the base of these data, the driver will be provided with several hints about how to improve his/her driving style, until he/she decides to end the trip
\end{itemize}
When the trip ends, the application will provide an after trip report graphic showing the followed route on a custom map.\\	
The application is comprised of two main features:

\begin{enumerate}
	\item \textbf{After Trip data presentation:} The user can inspect the details of the trip he/she just completed. In this screen the user can see:
	\begin{itemize}
		\item The route he/she followed, shown in a custom Google maps widget
		\item A general report about the completed trip providing the departure, the arrival, the date of the trip, the kilometres traveled, the time that the trip took and the \textit{Driver Safety Index} (DSI), that estimates the quality of the drive
	\end{itemize}
	\item \textbf{During Trip functionalities:} This functionality, is provided by the core \textit{Safecar's} algorithm. This algorithm is an engine that has the role of estimating the \textit{Driver Safety Index} based on navigation data coming from the car. These data come from a smart object, a plug, that has to be inserted in the custom car gate and has the capability of sending cleaned and custom driving style data to the connected device
\end{enumerate}
Based on these data, the algorithm computes the \textit{Driver Safety Index} (DSI). The engine architecture is easily exchangeable, given that the application is built in a parametric way. Building a more complex algorithm is only a matter of replacing the specific piece of code (a method) with another one that takes as input the same data and generates a same type index, but by implementing a different, maybe more complex, logic.
In order to provide the user with an almost continuous feedback about his/her driving style, this engine recomputes the DSI index once every 30 seconds.
This computation is done in a dedicated, separate thread that generates the current piece of advice to be sent to the user by simply switching on the value of the DSI. This hint is conveyed by filling a specific screen on the application.

\clearpage
\section{System constraints}
These are the constraints which must be met in order to allow the application to work correctly:

\begin{itemize}
	\item The user of the mobile should have partial internet access:
	\begin{itemize}
		\item The user application must have the internet access at least during the login and the logout procedure. In fact, during the actual use of the application it doesn't need the data connection
	\end{itemize}
	\item User's device should support bluetooth:
	\begin{itemize}
		\item The application needs the bluetooth for the interactive part, the one in which the app follows the user during the navigation. During the offline part, it doesn't need any bluetooth capability
	\end{itemize}
	\item User's device should support geolocalization:
	\begin{itemize}
		\item The application needs the geolocalization for the interactive part, the one in which the app follows the user during the navigation. During the offline part, it doesn't need any geolocalization capability
	\end{itemize}
\end{itemize}
\clearpage

\chapter{Architectural Design} \label{chap3}
The System Architecture is a way to give the overall view of a system and to put it in relation to external systems. This allows the reader to have a more complete and general idea of the entire system and at the same time to have a deeper view of the principal components of the system itself.
\clearpage

\chapter{User Interface Design} \label{chap4}
Here are presented some mockups that represent an idea of the structure of the application pages.

\section{Splash Screen and Registration \& Login Pages}
These mockups show an idea of the \textit{Splash Screen} and \textit{Registration \& Login} pages of the application. The Registration and Login Page allow the user of the application to register as a user of the service and this can be performed via the custom application functionality or via external providers.\\

\begin{figure}[!htbp]
  \centering
  \begin{minipage}[b]{0.45\textwidth}
    \includegraphics[width=\textwidth]{cpt/img/SplashScreen.png}
    \caption{Splash Screen}
  \end{minipage}
  \hfill
  \begin{minipage}[b]{0.45\textwidth}
    \includegraphics[width=\textwidth]{cpt/img/Login.png}
    \caption{Registration \& Login}
  \end{minipage}
\end{figure}

\clearpage
\section{Home Page}
These mockups show an idea of the \textit{Home Page} of the application. The Home Page presents a \textit{TabView} through which the user can navigate in order to inspect his history trips and a \textit{Navigation Drawer Menu} through which the user can reach other pages of the application.\\

\begin{figure}[htbp]
\centering
\includegraphics[width=\textwidth]{cpt/img/HomePage.png}
\caption{Home Page}
\end{figure}

\clearpage
\section{During Trip Page}
These mockups show an idea of the \textit{During Trip Page} of the application. The During Trip Page presents a \textit{View} where \textit{Hints} produced by the application are loaded and viewable by the user. The user can interact with the application using two buttons: the first one is a \textit{Pause/Resume} button which allow the user to pause the trip, without stopping it, if he wants to take a break from the driving session and, then, resume it; the second one, instead, is a \textit{Stop} button which allow the user to end the driving experience.\\

\begin{figure}[htbp]
\centering
\includegraphics[width=\textwidth]{cpt/img/DuringTrip.png}
\caption{During Trip}
\end{figure}

\clearpage
\section{Report Page}
This mockup show an idea of the \textit{Report Page} of the application. The Report Page is showed when the user inspect one of his history trips or when he complete a driving session pushing the \textit{Stop} button. This page allow the user to see all the details about his trip including data like: the trip's date, duration and length, the DSI score and the path he made.\\

\begin{figure}[htbp]
\centering
\includegraphics[width=0.7\textwidth]{cpt/img/ReportPage.png}
\caption{Report Page}
\end{figure}

\clearpage
\section{Profile Page}
These mockups show an idea of the \textit{Profile Page} of the application. The Profile Page shows information about the user like his name, surname, email address, driver level and also some badges that he can unlock meeting specific conditions. Each badge is clickable and let the user know about its locked/unlocked status.\\

\begin{figure}[htbp]
\centering
\includegraphics[width=\textwidth]{cpt/img/ProfilePage.png}
\caption{Profile Page}
\end{figure}

\clearpage
\section{Settings Page}
This mockup show an idea of the \textit{Settings Page} of the application. The Settings Page allow the user to manage some settings of the application like push notifications and smart objects. This page also allow the user to see some info about the application and send feedback in order to improve it.\\

\begin{figure}[htbp]
\centering
\includegraphics[width=0.55\textwidth]{cpt/img/SettingsPage.png}
\caption{Settings Page}
\end{figure}

\clearpage
\section{Smart Objects Page}
These mockups show an idea of the \textit{Smart Objects Pages} of the application. The Smart Objects Pages allow the user to manage smart objects which are necessary for the application. Using a bluetooth scanner the user can pair his device with a plug used to retrieve relevant information about the current driving session and he can also manage all the paired plugs, inspecting or deleting them.\\

\begin{figure}[htbp]
\centering
\includegraphics[width=\textwidth]{cpt/img/SmartObjectsPage.png}
\caption{Smart Objects Pages}
\end{figure}
\clearpage

\chapter{Future developments} \label{chap5}
The application can be improved by integrating the use of a \textit{text2speech} service that should have the role of reading the hints and send the relative audio to a speaker. This way the user will hear the hints during the drive.
\clearpage

\end{document}
